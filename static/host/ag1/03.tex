\documentclass{article}
\title{AG1 - Domácí zábava III}
\author{Tommy Chu}
\date{}

\usepackage[czech]{babel}
\usepackage[utf8]{inputenc}
\usepackage[T1]{fontenc}
\usepackage{graphicx}
\usepackage{amsmath}
\usepackage{amsthm}
\usepackage{amssymb}
\usepackage{subfiles}
\usepackage{hyperref}
\usepackage{geometry}
\usepackage{mathtools}
\usepackage{algpseudocode}
\usepackage{algorithm}
\usepackage{physics}
\usepackage{dsfont}
\usepackage{bm}
\usepackage{bbm}
\usepackage{float}
\usepackage{enumitem}
\usepackage{multirow}
\usepackage{tikz}
\usepackage{tikz-cd}
\usepackage{pgfplots}
\usepackage{lmodern}
\usepackage{import}
\usepackage{microtype}
\usepackage{fancyhdr}
\usepackage{parskip}
\usepackage{mystyle}
\usepackage{soul}

\newcommand{\mathcolorbox}[2]{\colorbox{#1}{$\displaystyle #2$}}
\definecolor{grey}{rgb}{.93,.93,.93}

\begin{document}
\maketitle

\section{Zadání}

Nechť $G = (V, E)$ je neorientovaný graf s $c$ komponentami souvislosti. Ukažte, že $G$ obsahuje alespoň $m - n + c$ kružnic.

\subsection*{Lemmata}

\paragraph*{Věta 1.} Přidáním nové hrany do souvislého grafu vznikne alespoň jedna nová kružnice.

\textit{Důkaz.} Protože je graf souvislý, mezi libovolnými dvěma vrcholy existuje cesta. Přidáním nové hrany vznikne mezi koncovými vrcholy nová cesta (délky 1). Slepením nové cesty a existující cesty, vznikne kružnice, která v původním grafu bez přidané hrany nebyla.
\qed

\paragraph*{Věta 2.} Souvislý graf má alespoň $m - n + 1$ kružnic.

\textit{Důkaz.}
Je-li graf strom, platí $m = n - 1$, a graf má triviálně alespoň $0 = m - n + 1$ kružnic. V opačném případě má graf díky souvislosti kostru z $n-1$ hran a jedná se o rozšíření této kostry o~$m - (n - 1) = m - n + 1$ hran. Nicméně v důsledku \textit{Věty 1} má strom rozšířený o $m - n + 1$ hran alespoň $m - n + 1$ kružnic.
\qed

\subsection*{Řešení}

Nechť $G = (V, E)$ je neorientovaný graf s $c$ komponentami souvislosti: $A_1, A_2, \cdots, A_c$. Označme $n_i = |V(A_i)|$ a $m_i = |E(A_i)|$. Protože jsou komponenty souvislé, z \textit{Věty 2} platí:
\[
    \text{pro $\forall i \in \{1, 2, ..., c\}$} \colon A_i \text{ má alespoň } m_i - n_i + 1 \text{ kružnic}.
\]

Z toho plyne, že počet kružnic grafu $G$ je alespoň
\begin{align*}
    \sum_{i=1}^{c} \Bigl( m_i - n_i + 1 \Bigr) = \sum_{i=1}^{c} m_i - \sum_{i=1}^{c} n_i + \sum_{i=1}^{c} 1 = m - n + c.
\end{align*}
\qed

\end{document}
