\section{Shluková analýza}

Shluková analýza (clustering) je metoda nesupervizovaného učení, která rozřadí data do nějakých shluků tak, aby byly blízké body v~jednom shluku a vzdálené v~různých. Jak přesně formalizovat takovou úlohu je problematické, proto může mít různá řešení v~závislosti na tom, jak definujeme vzdálenost.

\paragraph{Vzdálenost} Pro připomenutí, definice metriky byla zevedena v~sekci \ref{sec:metrika}.

\paragraph{Formalizace úlohy shlukování} Úloha na vstupu dostane metrický prostor $\calX$ se vzdáleností $d$, množinu dat $\D \subset \calX$ a počet požadovaných shluků $k$. Na výstupu se požaduje nějaký rozklad množiny $\D$ na $k$ shluků. To znamená $C_1, \ldots, C_k \subset \D$, kde $C_i \neq C_j$ pro všechna $i \neq j$, přičemž \[\D = \bigcup_{i=1}^{k} C_i.\]

\subsection{Hierarchické shlukování}

Hierarchické shlukování používá hladový aglomerativní přístup popsaný v~následujícím algoritmu.

\begin{algorithm}[H]
    \renewcommand{\thealgorithm}{}
    \caption{Hierarchické shlukování}
    \begin{algorithmic}[1]
        \Require
        \Statex Množina dat $\D$ a metrika pro vzdálenost shluků.
        \Statex Požadovaný počet shluků $k$ (jinak $k=1$).
        \Statex
        \State Uvažuj každý bod jako jednoprvkový shluk (celkem $|\D|$ shluků).
        \While{počet shluků $> k$}
        \State Nalezni dva nejbližší shluky.
        \State Tyto dva shluky spoj do jednoho.
        \EndWhile
    \end{algorithmic}
\end{algorithm}

\subsubsection{Měření vzdáleností shluků}

Po volbě metriky dvou bodů $d(a, b)$ je ještě nutné zvollit vzdálenost dvou shluků $D(A, B)$. Obvykle se použije jedna z~následujících:
\begin{itemize}

    \item Single linkage, který generuje dlouhé řetězce
          \[
              D(A, B) = \min_{\x \in A,\y \in B} d(\x, \y).
          \]

    \item Complete linkage, který generuje kompaktní buňky
          \[
              D(A, B) = \max_{\x \in A,\y \in B} d(\x, \y).
          \]

    \item Average linkage, který měří podle průměrné vzdálenosti všech bodů
          \[
              D(A, B) = \frac{1}{|A| |B|} \sum_{\x \in A,\y \in B} d(\x, \y).
          \]

    \item Wardova metoda (pouze pro $\calX = \R^p$), která minimalizuje nárůst vnitřního rozptylu buněk
          \[
              D(A, B) = \sum_{\x \in A \cup B} \|\x - \bm{\bar{x}}_{A \cup B} \|^2
              - \sum_{\x \in A} \|\x - \bm{\bar{x}}_A \|^2
              - \sum_{\x \in B} \|\x - \bm{\bar{x}}_B \|^2,
          \]
          kde $\bm{\bar{x}}_S$ značí geometrický střed (centroid) shluku $S$.

\end{itemize}

\subsubsection{Dendrogram}

Celý proces aglomerativního shlukování lze reprezentovat dendrogramem. To je strom, který má v~listech počáteční jednoprvkové shluky a ve vrcholech spojení. Obvykle algoritmus běží, dokud nespojí všechny shluky, v~takovém případě má v~kořeni finální shluk. Strom se vykresluje tak, že listy jsou ve výšce 0 a jednotlivé vrcholy ve výšce podle vzdálenosti shluků, které se v~daném bodě spojily (vzdálenost, kterou musely ``překonat'').

\begin{center}
    \scalebox{0.8}{%% Creator: Matplotlib, PGF backend
%%
%% To include the figure in your LaTeX document, write
%%   \input{<filename>.pgf}
%%
%% Make sure the required packages are loaded in your preamble
%%   \usepackage{pgf}
%%
%% Also ensure that all the required font packages are loaded; for instance,
%% the lmodern package is sometimes necessary when using math font.
%%   \usepackage{lmodern}
%%
%% Figures using additional raster images can only be included by \input if
%% they are in the same directory as the main LaTeX file. For loading figures
%% from other directories you can use the `import` package
%%   \usepackage{import}
%%
%% and then include the figures with
%%   \import{<path to file>}{<filename>.pgf}
%%
%% Matplotlib used the following preamble
%%   \def\mathdefault#1{#1}
%%   \everymath=\expandafter{\the\everymath\displaystyle}
%%   
%%   \usepackage{fontspec}
%%   \setmainfont{DejaVuSerif.ttf}[Path=\detokenize{/home/chutommy/miniconda3/envs/viz/lib/python3.10/site-packages/matplotlib/mpl-data/fonts/ttf/}]
%%   \setsansfont{DejaVuSans.ttf}[Path=\detokenize{/home/chutommy/miniconda3/envs/viz/lib/python3.10/site-packages/matplotlib/mpl-data/fonts/ttf/}]
%%   \setmonofont{DejaVuSansMono.ttf}[Path=\detokenize{/home/chutommy/miniconda3/envs/viz/lib/python3.10/site-packages/matplotlib/mpl-data/fonts/ttf/}]
%%   \makeatletter\@ifpackageloaded{underscore}{}{\usepackage[strings]{underscore}}\makeatother
%%
\begingroup%
\makeatletter%
\begin{pgfpicture}%
\pgfpathrectangle{\pgfpointorigin}{\pgfqpoint{8.000000in}{5.000000in}}%
\pgfusepath{use as bounding box, clip}%
\begin{pgfscope}%
\pgfsetbuttcap%
\pgfsetmiterjoin%
\definecolor{currentfill}{rgb}{1.000000,1.000000,1.000000}%
\pgfsetfillcolor{currentfill}%
\pgfsetlinewidth{0.000000pt}%
\definecolor{currentstroke}{rgb}{1.000000,1.000000,1.000000}%
\pgfsetstrokecolor{currentstroke}%
\pgfsetdash{}{0pt}%
\pgfpathmoveto{\pgfqpoint{0.000000in}{0.000000in}}%
\pgfpathlineto{\pgfqpoint{8.000000in}{0.000000in}}%
\pgfpathlineto{\pgfqpoint{8.000000in}{5.000000in}}%
\pgfpathlineto{\pgfqpoint{0.000000in}{5.000000in}}%
\pgfpathlineto{\pgfqpoint{0.000000in}{0.000000in}}%
\pgfpathclose%
\pgfusepath{fill}%
\end{pgfscope}%
\begin{pgfscope}%
\pgfsetbuttcap%
\pgfsetmiterjoin%
\definecolor{currentfill}{rgb}{1.000000,1.000000,1.000000}%
\pgfsetfillcolor{currentfill}%
\pgfsetlinewidth{0.000000pt}%
\definecolor{currentstroke}{rgb}{0.000000,0.000000,0.000000}%
\pgfsetstrokecolor{currentstroke}%
\pgfsetstrokeopacity{0.000000}%
\pgfsetdash{}{0pt}%
\pgfpathmoveto{\pgfqpoint{1.000000in}{0.550000in}}%
\pgfpathlineto{\pgfqpoint{7.200000in}{0.550000in}}%
\pgfpathlineto{\pgfqpoint{7.200000in}{4.400000in}}%
\pgfpathlineto{\pgfqpoint{1.000000in}{4.400000in}}%
\pgfpathlineto{\pgfqpoint{1.000000in}{0.550000in}}%
\pgfpathclose%
\pgfusepath{fill}%
\end{pgfscope}%
\begin{pgfscope}%
\definecolor{textcolor}{rgb}{0.000000,0.000000,0.000000}%
\pgfsetstrokecolor{textcolor}%
\pgfsetfillcolor{textcolor}%
\pgftext[x=1.193750in,y=0.452778in,,top]{\color{textcolor}{\sffamily\fontsize{12.000000}{14.400000}\selectfont\catcode`\^=\active\def^{\ifmmode\sp\else\^{}\fi}\catcode`\%=\active\def%{\%}(7)}}%
\end{pgfscope}%
\begin{pgfscope}%
\definecolor{textcolor}{rgb}{0.000000,0.000000,0.000000}%
\pgfsetstrokecolor{textcolor}%
\pgfsetfillcolor{textcolor}%
\pgftext[x=1.581250in,y=0.452778in,,top]{\color{textcolor}{\sffamily\fontsize{12.000000}{14.400000}\selectfont\catcode`\^=\active\def^{\ifmmode\sp\else\^{}\fi}\catcode`\%=\active\def%{\%}(8)}}%
\end{pgfscope}%
\begin{pgfscope}%
\definecolor{textcolor}{rgb}{0.000000,0.000000,0.000000}%
\pgfsetstrokecolor{textcolor}%
\pgfsetfillcolor{textcolor}%
\pgftext[x=1.968750in,y=0.452778in,,top]{\color{textcolor}{\sffamily\fontsize{12.000000}{14.400000}\selectfont\catcode`\^=\active\def^{\ifmmode\sp\else\^{}\fi}\catcode`\%=\active\def%{\%}41}}%
\end{pgfscope}%
\begin{pgfscope}%
\definecolor{textcolor}{rgb}{0.000000,0.000000,0.000000}%
\pgfsetstrokecolor{textcolor}%
\pgfsetfillcolor{textcolor}%
\pgftext[x=2.356250in,y=0.452778in,,top]{\color{textcolor}{\sffamily\fontsize{12.000000}{14.400000}\selectfont\catcode`\^=\active\def^{\ifmmode\sp\else\^{}\fi}\catcode`\%=\active\def%{\%}(5)}}%
\end{pgfscope}%
\begin{pgfscope}%
\definecolor{textcolor}{rgb}{0.000000,0.000000,0.000000}%
\pgfsetstrokecolor{textcolor}%
\pgfsetfillcolor{textcolor}%
\pgftext[x=2.743750in,y=0.452778in,,top]{\color{textcolor}{\sffamily\fontsize{12.000000}{14.400000}\selectfont\catcode`\^=\active\def^{\ifmmode\sp\else\^{}\fi}\catcode`\%=\active\def%{\%}(10)}}%
\end{pgfscope}%
\begin{pgfscope}%
\definecolor{textcolor}{rgb}{0.000000,0.000000,0.000000}%
\pgfsetstrokecolor{textcolor}%
\pgfsetfillcolor{textcolor}%
\pgftext[x=3.131250in,y=0.452778in,,top]{\color{textcolor}{\sffamily\fontsize{12.000000}{14.400000}\selectfont\catcode`\^=\active\def^{\ifmmode\sp\else\^{}\fi}\catcode`\%=\active\def%{\%}(7)}}%
\end{pgfscope}%
\begin{pgfscope}%
\definecolor{textcolor}{rgb}{0.000000,0.000000,0.000000}%
\pgfsetstrokecolor{textcolor}%
\pgfsetfillcolor{textcolor}%
\pgftext[x=3.518750in,y=0.452778in,,top]{\color{textcolor}{\sffamily\fontsize{12.000000}{14.400000}\selectfont\catcode`\^=\active\def^{\ifmmode\sp\else\^{}\fi}\catcode`\%=\active\def%{\%}(4)}}%
\end{pgfscope}%
\begin{pgfscope}%
\definecolor{textcolor}{rgb}{0.000000,0.000000,0.000000}%
\pgfsetstrokecolor{textcolor}%
\pgfsetfillcolor{textcolor}%
\pgftext[x=3.906250in,y=0.452778in,,top]{\color{textcolor}{\sffamily\fontsize{12.000000}{14.400000}\selectfont\catcode`\^=\active\def^{\ifmmode\sp\else\^{}\fi}\catcode`\%=\active\def%{\%}(8)}}%
\end{pgfscope}%
\begin{pgfscope}%
\definecolor{textcolor}{rgb}{0.000000,0.000000,0.000000}%
\pgfsetstrokecolor{textcolor}%
\pgfsetfillcolor{textcolor}%
\pgftext[x=4.293750in,y=0.452778in,,top]{\color{textcolor}{\sffamily\fontsize{12.000000}{14.400000}\selectfont\catcode`\^=\active\def^{\ifmmode\sp\else\^{}\fi}\catcode`\%=\active\def%{\%}(9)}}%
\end{pgfscope}%
\begin{pgfscope}%
\definecolor{textcolor}{rgb}{0.000000,0.000000,0.000000}%
\pgfsetstrokecolor{textcolor}%
\pgfsetfillcolor{textcolor}%
\pgftext[x=4.681250in,y=0.452778in,,top]{\color{textcolor}{\sffamily\fontsize{12.000000}{14.400000}\selectfont\catcode`\^=\active\def^{\ifmmode\sp\else\^{}\fi}\catcode`\%=\active\def%{\%}(15)}}%
\end{pgfscope}%
\begin{pgfscope}%
\definecolor{textcolor}{rgb}{0.000000,0.000000,0.000000}%
\pgfsetstrokecolor{textcolor}%
\pgfsetfillcolor{textcolor}%
\pgftext[x=5.068750in,y=0.452778in,,top]{\color{textcolor}{\sffamily\fontsize{12.000000}{14.400000}\selectfont\catcode`\^=\active\def^{\ifmmode\sp\else\^{}\fi}\catcode`\%=\active\def%{\%}(5)}}%
\end{pgfscope}%
\begin{pgfscope}%
\definecolor{textcolor}{rgb}{0.000000,0.000000,0.000000}%
\pgfsetstrokecolor{textcolor}%
\pgfsetfillcolor{textcolor}%
\pgftext[x=5.456250in,y=0.452778in,,top]{\color{textcolor}{\sffamily\fontsize{12.000000}{14.400000}\selectfont\catcode`\^=\active\def^{\ifmmode\sp\else\^{}\fi}\catcode`\%=\active\def%{\%}(7)}}%
\end{pgfscope}%
\begin{pgfscope}%
\definecolor{textcolor}{rgb}{0.000000,0.000000,0.000000}%
\pgfsetstrokecolor{textcolor}%
\pgfsetfillcolor{textcolor}%
\pgftext[x=5.843750in,y=0.452778in,,top]{\color{textcolor}{\sffamily\fontsize{12.000000}{14.400000}\selectfont\catcode`\^=\active\def^{\ifmmode\sp\else\^{}\fi}\catcode`\%=\active\def%{\%}(4)}}%
\end{pgfscope}%
\begin{pgfscope}%
\definecolor{textcolor}{rgb}{0.000000,0.000000,0.000000}%
\pgfsetstrokecolor{textcolor}%
\pgfsetfillcolor{textcolor}%
\pgftext[x=6.231250in,y=0.452778in,,top]{\color{textcolor}{\sffamily\fontsize{12.000000}{14.400000}\selectfont\catcode`\^=\active\def^{\ifmmode\sp\else\^{}\fi}\catcode`\%=\active\def%{\%}(22)}}%
\end{pgfscope}%
\begin{pgfscope}%
\definecolor{textcolor}{rgb}{0.000000,0.000000,0.000000}%
\pgfsetstrokecolor{textcolor}%
\pgfsetfillcolor{textcolor}%
\pgftext[x=6.618750in,y=0.452778in,,top]{\color{textcolor}{\sffamily\fontsize{12.000000}{14.400000}\selectfont\catcode`\^=\active\def^{\ifmmode\sp\else\^{}\fi}\catcode`\%=\active\def%{\%}(15)}}%
\end{pgfscope}%
\begin{pgfscope}%
\definecolor{textcolor}{rgb}{0.000000,0.000000,0.000000}%
\pgfsetstrokecolor{textcolor}%
\pgfsetfillcolor{textcolor}%
\pgftext[x=7.006250in,y=0.452778in,,top]{\color{textcolor}{\sffamily\fontsize{12.000000}{14.400000}\selectfont\catcode`\^=\active\def^{\ifmmode\sp\else\^{}\fi}\catcode`\%=\active\def%{\%}(23)}}%
\end{pgfscope}%
\begin{pgfscope}%
\pgfsetbuttcap%
\pgfsetroundjoin%
\definecolor{currentfill}{rgb}{0.000000,0.000000,0.000000}%
\pgfsetfillcolor{currentfill}%
\pgfsetlinewidth{0.803000pt}%
\definecolor{currentstroke}{rgb}{0.000000,0.000000,0.000000}%
\pgfsetstrokecolor{currentstroke}%
\pgfsetdash{}{0pt}%
\pgfsys@defobject{currentmarker}{\pgfqpoint{-0.048611in}{0.000000in}}{\pgfqpoint{-0.000000in}{0.000000in}}{%
\pgfpathmoveto{\pgfqpoint{-0.000000in}{0.000000in}}%
\pgfpathlineto{\pgfqpoint{-0.048611in}{0.000000in}}%
\pgfusepath{stroke,fill}%
}%
\begin{pgfscope}%
\pgfsys@transformshift{1.000000in}{0.550000in}%
\pgfsys@useobject{currentmarker}{}%
\end{pgfscope}%
\end{pgfscope}%
\begin{pgfscope}%
\definecolor{textcolor}{rgb}{0.000000,0.000000,0.000000}%
\pgfsetstrokecolor{textcolor}%
\pgfsetfillcolor{textcolor}%
\pgftext[x=0.814412in, y=0.497238in, left, base]{\color{textcolor}{\sffamily\fontsize{10.000000}{12.000000}\selectfont\catcode`\^=\active\def^{\ifmmode\sp\else\^{}\fi}\catcode`\%=\active\def%{\%}0}}%
\end{pgfscope}%
\begin{pgfscope}%
\pgfsetbuttcap%
\pgfsetroundjoin%
\definecolor{currentfill}{rgb}{0.000000,0.000000,0.000000}%
\pgfsetfillcolor{currentfill}%
\pgfsetlinewidth{0.803000pt}%
\definecolor{currentstroke}{rgb}{0.000000,0.000000,0.000000}%
\pgfsetstrokecolor{currentstroke}%
\pgfsetdash{}{0pt}%
\pgfsys@defobject{currentmarker}{\pgfqpoint{-0.048611in}{0.000000in}}{\pgfqpoint{-0.000000in}{0.000000in}}{%
\pgfpathmoveto{\pgfqpoint{-0.000000in}{0.000000in}}%
\pgfpathlineto{\pgfqpoint{-0.048611in}{0.000000in}}%
\pgfusepath{stroke,fill}%
}%
\begin{pgfscope}%
\pgfsys@transformshift{1.000000in}{1.115013in}%
\pgfsys@useobject{currentmarker}{}%
\end{pgfscope}%
\end{pgfscope}%
\begin{pgfscope}%
\definecolor{textcolor}{rgb}{0.000000,0.000000,0.000000}%
\pgfsetstrokecolor{textcolor}%
\pgfsetfillcolor{textcolor}%
\pgftext[x=0.814412in, y=1.062252in, left, base]{\color{textcolor}{\sffamily\fontsize{10.000000}{12.000000}\selectfont\catcode`\^=\active\def^{\ifmmode\sp\else\^{}\fi}\catcode`\%=\active\def%{\%}5}}%
\end{pgfscope}%
\begin{pgfscope}%
\pgfsetbuttcap%
\pgfsetroundjoin%
\definecolor{currentfill}{rgb}{0.000000,0.000000,0.000000}%
\pgfsetfillcolor{currentfill}%
\pgfsetlinewidth{0.803000pt}%
\definecolor{currentstroke}{rgb}{0.000000,0.000000,0.000000}%
\pgfsetstrokecolor{currentstroke}%
\pgfsetdash{}{0pt}%
\pgfsys@defobject{currentmarker}{\pgfqpoint{-0.048611in}{0.000000in}}{\pgfqpoint{-0.000000in}{0.000000in}}{%
\pgfpathmoveto{\pgfqpoint{-0.000000in}{0.000000in}}%
\pgfpathlineto{\pgfqpoint{-0.048611in}{0.000000in}}%
\pgfusepath{stroke,fill}%
}%
\begin{pgfscope}%
\pgfsys@transformshift{1.000000in}{1.680027in}%
\pgfsys@useobject{currentmarker}{}%
\end{pgfscope}%
\end{pgfscope}%
\begin{pgfscope}%
\definecolor{textcolor}{rgb}{0.000000,0.000000,0.000000}%
\pgfsetstrokecolor{textcolor}%
\pgfsetfillcolor{textcolor}%
\pgftext[x=0.726047in, y=1.627265in, left, base]{\color{textcolor}{\sffamily\fontsize{10.000000}{12.000000}\selectfont\catcode`\^=\active\def^{\ifmmode\sp\else\^{}\fi}\catcode`\%=\active\def%{\%}10}}%
\end{pgfscope}%
\begin{pgfscope}%
\pgfsetbuttcap%
\pgfsetroundjoin%
\definecolor{currentfill}{rgb}{0.000000,0.000000,0.000000}%
\pgfsetfillcolor{currentfill}%
\pgfsetlinewidth{0.803000pt}%
\definecolor{currentstroke}{rgb}{0.000000,0.000000,0.000000}%
\pgfsetstrokecolor{currentstroke}%
\pgfsetdash{}{0pt}%
\pgfsys@defobject{currentmarker}{\pgfqpoint{-0.048611in}{0.000000in}}{\pgfqpoint{-0.000000in}{0.000000in}}{%
\pgfpathmoveto{\pgfqpoint{-0.000000in}{0.000000in}}%
\pgfpathlineto{\pgfqpoint{-0.048611in}{0.000000in}}%
\pgfusepath{stroke,fill}%
}%
\begin{pgfscope}%
\pgfsys@transformshift{1.000000in}{2.245040in}%
\pgfsys@useobject{currentmarker}{}%
\end{pgfscope}%
\end{pgfscope}%
\begin{pgfscope}%
\definecolor{textcolor}{rgb}{0.000000,0.000000,0.000000}%
\pgfsetstrokecolor{textcolor}%
\pgfsetfillcolor{textcolor}%
\pgftext[x=0.726047in, y=2.192279in, left, base]{\color{textcolor}{\sffamily\fontsize{10.000000}{12.000000}\selectfont\catcode`\^=\active\def^{\ifmmode\sp\else\^{}\fi}\catcode`\%=\active\def%{\%}15}}%
\end{pgfscope}%
\begin{pgfscope}%
\pgfsetbuttcap%
\pgfsetroundjoin%
\definecolor{currentfill}{rgb}{0.000000,0.000000,0.000000}%
\pgfsetfillcolor{currentfill}%
\pgfsetlinewidth{0.803000pt}%
\definecolor{currentstroke}{rgb}{0.000000,0.000000,0.000000}%
\pgfsetstrokecolor{currentstroke}%
\pgfsetdash{}{0pt}%
\pgfsys@defobject{currentmarker}{\pgfqpoint{-0.048611in}{0.000000in}}{\pgfqpoint{-0.000000in}{0.000000in}}{%
\pgfpathmoveto{\pgfqpoint{-0.000000in}{0.000000in}}%
\pgfpathlineto{\pgfqpoint{-0.048611in}{0.000000in}}%
\pgfusepath{stroke,fill}%
}%
\begin{pgfscope}%
\pgfsys@transformshift{1.000000in}{2.810054in}%
\pgfsys@useobject{currentmarker}{}%
\end{pgfscope}%
\end{pgfscope}%
\begin{pgfscope}%
\definecolor{textcolor}{rgb}{0.000000,0.000000,0.000000}%
\pgfsetstrokecolor{textcolor}%
\pgfsetfillcolor{textcolor}%
\pgftext[x=0.726047in, y=2.757292in, left, base]{\color{textcolor}{\sffamily\fontsize{10.000000}{12.000000}\selectfont\catcode`\^=\active\def^{\ifmmode\sp\else\^{}\fi}\catcode`\%=\active\def%{\%}20}}%
\end{pgfscope}%
\begin{pgfscope}%
\pgfsetbuttcap%
\pgfsetroundjoin%
\definecolor{currentfill}{rgb}{0.000000,0.000000,0.000000}%
\pgfsetfillcolor{currentfill}%
\pgfsetlinewidth{0.803000pt}%
\definecolor{currentstroke}{rgb}{0.000000,0.000000,0.000000}%
\pgfsetstrokecolor{currentstroke}%
\pgfsetdash{}{0pt}%
\pgfsys@defobject{currentmarker}{\pgfqpoint{-0.048611in}{0.000000in}}{\pgfqpoint{-0.000000in}{0.000000in}}{%
\pgfpathmoveto{\pgfqpoint{-0.000000in}{0.000000in}}%
\pgfpathlineto{\pgfqpoint{-0.048611in}{0.000000in}}%
\pgfusepath{stroke,fill}%
}%
\begin{pgfscope}%
\pgfsys@transformshift{1.000000in}{3.375067in}%
\pgfsys@useobject{currentmarker}{}%
\end{pgfscope}%
\end{pgfscope}%
\begin{pgfscope}%
\definecolor{textcolor}{rgb}{0.000000,0.000000,0.000000}%
\pgfsetstrokecolor{textcolor}%
\pgfsetfillcolor{textcolor}%
\pgftext[x=0.726047in, y=3.322306in, left, base]{\color{textcolor}{\sffamily\fontsize{10.000000}{12.000000}\selectfont\catcode`\^=\active\def^{\ifmmode\sp\else\^{}\fi}\catcode`\%=\active\def%{\%}25}}%
\end{pgfscope}%
\begin{pgfscope}%
\pgfsetbuttcap%
\pgfsetroundjoin%
\definecolor{currentfill}{rgb}{0.000000,0.000000,0.000000}%
\pgfsetfillcolor{currentfill}%
\pgfsetlinewidth{0.803000pt}%
\definecolor{currentstroke}{rgb}{0.000000,0.000000,0.000000}%
\pgfsetstrokecolor{currentstroke}%
\pgfsetdash{}{0pt}%
\pgfsys@defobject{currentmarker}{\pgfqpoint{-0.048611in}{0.000000in}}{\pgfqpoint{-0.000000in}{0.000000in}}{%
\pgfpathmoveto{\pgfqpoint{-0.000000in}{0.000000in}}%
\pgfpathlineto{\pgfqpoint{-0.048611in}{0.000000in}}%
\pgfusepath{stroke,fill}%
}%
\begin{pgfscope}%
\pgfsys@transformshift{1.000000in}{3.940081in}%
\pgfsys@useobject{currentmarker}{}%
\end{pgfscope}%
\end{pgfscope}%
\begin{pgfscope}%
\definecolor{textcolor}{rgb}{0.000000,0.000000,0.000000}%
\pgfsetstrokecolor{textcolor}%
\pgfsetfillcolor{textcolor}%
\pgftext[x=0.726047in, y=3.887319in, left, base]{\color{textcolor}{\sffamily\fontsize{10.000000}{12.000000}\selectfont\catcode`\^=\active\def^{\ifmmode\sp\else\^{}\fi}\catcode`\%=\active\def%{\%}30}}%
\end{pgfscope}%
\begin{pgfscope}%
\pgfpathrectangle{\pgfqpoint{1.000000in}{0.550000in}}{\pgfqpoint{6.200000in}{3.850000in}}%
\pgfusepath{clip}%
\pgfsetbuttcap%
\pgfsetroundjoin%
\pgfsetlinewidth{1.505625pt}%
\definecolor{currentstroke}{rgb}{1.000000,0.498039,0.054902}%
\pgfsetstrokecolor{currentstroke}%
\pgfsetdash{}{0pt}%
\pgfpathmoveto{\pgfqpoint{1.193750in}{0.550000in}}%
\pgfpathlineto{\pgfqpoint{1.193750in}{0.642520in}}%
\pgfpathlineto{\pgfqpoint{1.581250in}{0.642520in}}%
\pgfpathlineto{\pgfqpoint{1.581250in}{0.550000in}}%
\pgfusepath{stroke}%
\end{pgfscope}%
\begin{pgfscope}%
\pgfpathrectangle{\pgfqpoint{1.000000in}{0.550000in}}{\pgfqpoint{6.200000in}{3.850000in}}%
\pgfusepath{clip}%
\pgfsetbuttcap%
\pgfsetroundjoin%
\pgfsetlinewidth{1.505625pt}%
\definecolor{currentstroke}{rgb}{1.000000,0.498039,0.054902}%
\pgfsetstrokecolor{currentstroke}%
\pgfsetdash{}{0pt}%
\pgfpathmoveto{\pgfqpoint{1.968750in}{0.550000in}}%
\pgfpathlineto{\pgfqpoint{1.968750in}{0.674884in}}%
\pgfpathlineto{\pgfqpoint{2.356250in}{0.674884in}}%
\pgfpathlineto{\pgfqpoint{2.356250in}{0.550000in}}%
\pgfusepath{stroke}%
\end{pgfscope}%
\begin{pgfscope}%
\pgfpathrectangle{\pgfqpoint{1.000000in}{0.550000in}}{\pgfqpoint{6.200000in}{3.850000in}}%
\pgfusepath{clip}%
\pgfsetbuttcap%
\pgfsetroundjoin%
\pgfsetlinewidth{1.505625pt}%
\definecolor{currentstroke}{rgb}{1.000000,0.498039,0.054902}%
\pgfsetstrokecolor{currentstroke}%
\pgfsetdash{}{0pt}%
\pgfpathmoveto{\pgfqpoint{1.387500in}{0.642520in}}%
\pgfpathlineto{\pgfqpoint{1.387500in}{0.704957in}}%
\pgfpathlineto{\pgfqpoint{2.162500in}{0.704957in}}%
\pgfpathlineto{\pgfqpoint{2.162500in}{0.674884in}}%
\pgfusepath{stroke}%
\end{pgfscope}%
\begin{pgfscope}%
\pgfpathrectangle{\pgfqpoint{1.000000in}{0.550000in}}{\pgfqpoint{6.200000in}{3.850000in}}%
\pgfusepath{clip}%
\pgfsetbuttcap%
\pgfsetroundjoin%
\pgfsetlinewidth{1.505625pt}%
\definecolor{currentstroke}{rgb}{1.000000,0.498039,0.054902}%
\pgfsetstrokecolor{currentstroke}%
\pgfsetdash{}{0pt}%
\pgfpathmoveto{\pgfqpoint{2.743750in}{0.550000in}}%
\pgfpathlineto{\pgfqpoint{2.743750in}{0.652830in}}%
\pgfpathlineto{\pgfqpoint{3.131250in}{0.652830in}}%
\pgfpathlineto{\pgfqpoint{3.131250in}{0.550000in}}%
\pgfusepath{stroke}%
\end{pgfscope}%
\begin{pgfscope}%
\pgfpathrectangle{\pgfqpoint{1.000000in}{0.550000in}}{\pgfqpoint{6.200000in}{3.850000in}}%
\pgfusepath{clip}%
\pgfsetbuttcap%
\pgfsetroundjoin%
\pgfsetlinewidth{1.505625pt}%
\definecolor{currentstroke}{rgb}{1.000000,0.498039,0.054902}%
\pgfsetstrokecolor{currentstroke}%
\pgfsetdash{}{0pt}%
\pgfpathmoveto{\pgfqpoint{3.518750in}{0.550000in}}%
\pgfpathlineto{\pgfqpoint{3.518750in}{0.691705in}}%
\pgfpathlineto{\pgfqpoint{3.906250in}{0.691705in}}%
\pgfpathlineto{\pgfqpoint{3.906250in}{0.550000in}}%
\pgfusepath{stroke}%
\end{pgfscope}%
\begin{pgfscope}%
\pgfpathrectangle{\pgfqpoint{1.000000in}{0.550000in}}{\pgfqpoint{6.200000in}{3.850000in}}%
\pgfusepath{clip}%
\pgfsetbuttcap%
\pgfsetroundjoin%
\pgfsetlinewidth{1.505625pt}%
\definecolor{currentstroke}{rgb}{1.000000,0.498039,0.054902}%
\pgfsetstrokecolor{currentstroke}%
\pgfsetdash{}{0pt}%
\pgfpathmoveto{\pgfqpoint{2.937500in}{0.652830in}}%
\pgfpathlineto{\pgfqpoint{2.937500in}{0.766824in}}%
\pgfpathlineto{\pgfqpoint{3.712500in}{0.766824in}}%
\pgfpathlineto{\pgfqpoint{3.712500in}{0.691705in}}%
\pgfusepath{stroke}%
\end{pgfscope}%
\begin{pgfscope}%
\pgfpathrectangle{\pgfqpoint{1.000000in}{0.550000in}}{\pgfqpoint{6.200000in}{3.850000in}}%
\pgfusepath{clip}%
\pgfsetbuttcap%
\pgfsetroundjoin%
\pgfsetlinewidth{1.505625pt}%
\definecolor{currentstroke}{rgb}{1.000000,0.498039,0.054902}%
\pgfsetstrokecolor{currentstroke}%
\pgfsetdash{}{0pt}%
\pgfpathmoveto{\pgfqpoint{1.775000in}{0.704957in}}%
\pgfpathlineto{\pgfqpoint{1.775000in}{0.982580in}}%
\pgfpathlineto{\pgfqpoint{3.325000in}{0.982580in}}%
\pgfpathlineto{\pgfqpoint{3.325000in}{0.766824in}}%
\pgfusepath{stroke}%
\end{pgfscope}%
\begin{pgfscope}%
\pgfpathrectangle{\pgfqpoint{1.000000in}{0.550000in}}{\pgfqpoint{6.200000in}{3.850000in}}%
\pgfusepath{clip}%
\pgfsetbuttcap%
\pgfsetroundjoin%
\pgfsetlinewidth{1.505625pt}%
\definecolor{currentstroke}{rgb}{0.172549,0.627451,0.172549}%
\pgfsetstrokecolor{currentstroke}%
\pgfsetdash{}{0pt}%
\pgfpathmoveto{\pgfqpoint{4.293750in}{0.550000in}}%
\pgfpathlineto{\pgfqpoint{4.293750in}{0.748935in}}%
\pgfpathlineto{\pgfqpoint{4.681250in}{0.748935in}}%
\pgfpathlineto{\pgfqpoint{4.681250in}{0.550000in}}%
\pgfusepath{stroke}%
\end{pgfscope}%
\begin{pgfscope}%
\pgfpathrectangle{\pgfqpoint{1.000000in}{0.550000in}}{\pgfqpoint{6.200000in}{3.850000in}}%
\pgfusepath{clip}%
\pgfsetbuttcap%
\pgfsetroundjoin%
\pgfsetlinewidth{1.505625pt}%
\definecolor{currentstroke}{rgb}{0.172549,0.627451,0.172549}%
\pgfsetstrokecolor{currentstroke}%
\pgfsetdash{}{0pt}%
\pgfpathmoveto{\pgfqpoint{5.068750in}{0.550000in}}%
\pgfpathlineto{\pgfqpoint{5.068750in}{0.758585in}}%
\pgfpathlineto{\pgfqpoint{5.456250in}{0.758585in}}%
\pgfpathlineto{\pgfqpoint{5.456250in}{0.550000in}}%
\pgfusepath{stroke}%
\end{pgfscope}%
\begin{pgfscope}%
\pgfpathrectangle{\pgfqpoint{1.000000in}{0.550000in}}{\pgfqpoint{6.200000in}{3.850000in}}%
\pgfusepath{clip}%
\pgfsetbuttcap%
\pgfsetroundjoin%
\pgfsetlinewidth{1.505625pt}%
\definecolor{currentstroke}{rgb}{0.172549,0.627451,0.172549}%
\pgfsetstrokecolor{currentstroke}%
\pgfsetdash{}{0pt}%
\pgfpathmoveto{\pgfqpoint{4.487500in}{0.748935in}}%
\pgfpathlineto{\pgfqpoint{4.487500in}{1.097804in}}%
\pgfpathlineto{\pgfqpoint{5.262500in}{1.097804in}}%
\pgfpathlineto{\pgfqpoint{5.262500in}{0.758585in}}%
\pgfusepath{stroke}%
\end{pgfscope}%
\begin{pgfscope}%
\pgfpathrectangle{\pgfqpoint{1.000000in}{0.550000in}}{\pgfqpoint{6.200000in}{3.850000in}}%
\pgfusepath{clip}%
\pgfsetbuttcap%
\pgfsetroundjoin%
\pgfsetlinewidth{1.505625pt}%
\definecolor{currentstroke}{rgb}{0.172549,0.627451,0.172549}%
\pgfsetstrokecolor{currentstroke}%
\pgfsetdash{}{0pt}%
\pgfpathmoveto{\pgfqpoint{5.843750in}{0.550000in}}%
\pgfpathlineto{\pgfqpoint{5.843750in}{0.867983in}}%
\pgfpathlineto{\pgfqpoint{6.231250in}{0.867983in}}%
\pgfpathlineto{\pgfqpoint{6.231250in}{0.550000in}}%
\pgfusepath{stroke}%
\end{pgfscope}%
\begin{pgfscope}%
\pgfpathrectangle{\pgfqpoint{1.000000in}{0.550000in}}{\pgfqpoint{6.200000in}{3.850000in}}%
\pgfusepath{clip}%
\pgfsetbuttcap%
\pgfsetroundjoin%
\pgfsetlinewidth{1.505625pt}%
\definecolor{currentstroke}{rgb}{0.172549,0.627451,0.172549}%
\pgfsetstrokecolor{currentstroke}%
\pgfsetdash{}{0pt}%
\pgfpathmoveto{\pgfqpoint{6.618750in}{0.550000in}}%
\pgfpathlineto{\pgfqpoint{6.618750in}{0.874252in}}%
\pgfpathlineto{\pgfqpoint{7.006250in}{0.874252in}}%
\pgfpathlineto{\pgfqpoint{7.006250in}{0.550000in}}%
\pgfusepath{stroke}%
\end{pgfscope}%
\begin{pgfscope}%
\pgfpathrectangle{\pgfqpoint{1.000000in}{0.550000in}}{\pgfqpoint{6.200000in}{3.850000in}}%
\pgfusepath{clip}%
\pgfsetbuttcap%
\pgfsetroundjoin%
\pgfsetlinewidth{1.505625pt}%
\definecolor{currentstroke}{rgb}{0.172549,0.627451,0.172549}%
\pgfsetstrokecolor{currentstroke}%
\pgfsetdash{}{0pt}%
\pgfpathmoveto{\pgfqpoint{6.037500in}{0.867983in}}%
\pgfpathlineto{\pgfqpoint{6.037500in}{1.273150in}}%
\pgfpathlineto{\pgfqpoint{6.812500in}{1.273150in}}%
\pgfpathlineto{\pgfqpoint{6.812500in}{0.874252in}}%
\pgfusepath{stroke}%
\end{pgfscope}%
\begin{pgfscope}%
\pgfpathrectangle{\pgfqpoint{1.000000in}{0.550000in}}{\pgfqpoint{6.200000in}{3.850000in}}%
\pgfusepath{clip}%
\pgfsetbuttcap%
\pgfsetroundjoin%
\pgfsetlinewidth{1.505625pt}%
\definecolor{currentstroke}{rgb}{0.172549,0.627451,0.172549}%
\pgfsetstrokecolor{currentstroke}%
\pgfsetdash{}{0pt}%
\pgfpathmoveto{\pgfqpoint{4.875000in}{1.097804in}}%
\pgfpathlineto{\pgfqpoint{4.875000in}{1.939978in}}%
\pgfpathlineto{\pgfqpoint{6.425000in}{1.939978in}}%
\pgfpathlineto{\pgfqpoint{6.425000in}{1.273150in}}%
\pgfusepath{stroke}%
\end{pgfscope}%
\begin{pgfscope}%
\pgfpathrectangle{\pgfqpoint{1.000000in}{0.550000in}}{\pgfqpoint{6.200000in}{3.850000in}}%
\pgfusepath{clip}%
\pgfsetbuttcap%
\pgfsetroundjoin%
\pgfsetlinewidth{1.505625pt}%
\definecolor{currentstroke}{rgb}{0.121569,0.466667,0.705882}%
\pgfsetstrokecolor{currentstroke}%
\pgfsetdash{}{0pt}%
\pgfpathmoveto{\pgfqpoint{2.550000in}{0.982580in}}%
\pgfpathlineto{\pgfqpoint{2.550000in}{4.216667in}}%
\pgfpathlineto{\pgfqpoint{5.650000in}{4.216667in}}%
\pgfpathlineto{\pgfqpoint{5.650000in}{1.939978in}}%
\pgfusepath{stroke}%
\end{pgfscope}%
\begin{pgfscope}%
\pgfsetrectcap%
\pgfsetmiterjoin%
\pgfsetlinewidth{0.803000pt}%
\definecolor{currentstroke}{rgb}{0.000000,0.000000,0.000000}%
\pgfsetstrokecolor{currentstroke}%
\pgfsetdash{}{0pt}%
\pgfpathmoveto{\pgfqpoint{1.000000in}{0.550000in}}%
\pgfpathlineto{\pgfqpoint{1.000000in}{4.400000in}}%
\pgfusepath{stroke}%
\end{pgfscope}%
\begin{pgfscope}%
\pgfsetrectcap%
\pgfsetmiterjoin%
\pgfsetlinewidth{0.803000pt}%
\definecolor{currentstroke}{rgb}{0.000000,0.000000,0.000000}%
\pgfsetstrokecolor{currentstroke}%
\pgfsetdash{}{0pt}%
\pgfpathmoveto{\pgfqpoint{7.200000in}{0.550000in}}%
\pgfpathlineto{\pgfqpoint{7.200000in}{4.400000in}}%
\pgfusepath{stroke}%
\end{pgfscope}%
\begin{pgfscope}%
\pgfsetrectcap%
\pgfsetmiterjoin%
\pgfsetlinewidth{0.803000pt}%
\definecolor{currentstroke}{rgb}{0.000000,0.000000,0.000000}%
\pgfsetstrokecolor{currentstroke}%
\pgfsetdash{}{0pt}%
\pgfpathmoveto{\pgfqpoint{1.000000in}{0.550000in}}%
\pgfpathlineto{\pgfqpoint{7.200000in}{0.550000in}}%
\pgfusepath{stroke}%
\end{pgfscope}%
\begin{pgfscope}%
\pgfsetrectcap%
\pgfsetmiterjoin%
\pgfsetlinewidth{0.803000pt}%
\definecolor{currentstroke}{rgb}{0.000000,0.000000,0.000000}%
\pgfsetstrokecolor{currentstroke}%
\pgfsetdash{}{0pt}%
\pgfpathmoveto{\pgfqpoint{1.000000in}{4.400000in}}%
\pgfpathlineto{\pgfqpoint{7.200000in}{4.400000in}}%
\pgfusepath{stroke}%
\end{pgfscope}%
\end{pgfpicture}%
\makeatother%
\endgroup%
}
\end{center}

Máme-li požadované $k$, příslušně dendrogram rozřízneme tak, aby vodorovná čára protla přesně $k$ hran. Každá taková hrana odpovídá nějakému shluku. Případně můžeme zvolit nějakou danou výšku (přijatelnou vzdálenost shluků).

\paragraph{Shrnuťí} Výhodou Hierarchického shlukování je flexibilita. Můžeme algoritmus provést a pak následně rozhodovat o~rozdělení. Při změně počtu shluků stačí jen změnit místo, ve kterém provádíme řez (struktura dendrogramu se nemění).

Nevýhodou hierarchického shlukování je výpočetní náročnost $\mathcal{O}(n^3)$, v~případě single/complete linkage lze zefektivnit na $\mathcal{O}(n^2)$. Pro velké datové soubory se proto nehodí.

\subsection{Algoritmus k-means}

\subsubsection{Shlukování jako optimalizační úloha}

Ke shlukování lze přistupovat jako k~optimalizační úloze, ve které minimalizujeme účelovou funkci (objective function).

Pro dané $k$ hledáme rozklad $C = (C_1, \ldots, C_k)$ množiny dat $\X$ v~metrickém prostoru $\calX = \R^p$ vybavenou Eukleidovskou vzdáleností minimalizující účelovou funkci
\[
    G(C) = \sum_{C_i} \frac{1}{2 |C_i|} \sum_{\x, \y \in C_i} \|\x - \y\|^2
\]
Jedná se o~průměrnou kvadratickou vzdálenost vnitřních bodů.

\paragraph{Souvislost účelové funkce s~geometrickým středem}
Lze ukázat, že takový součet lze vyjádřit jako součet kvadrátů vzdálenosti od geometrického středu příslušných shluků.
\[
    \frac{1}{2 |C_i|} \sum_{\x, \y \in C_i} \|\x - \y\|^2
    = \sum_{\x \in C_i} \|\x - \bm{\bar{x}}_{C_i}\|^2
    = \min_{\bmu \in \R^p} \sum_{\x \in C_i} \|\x - \bmu\|^2
\]

\subsubsection{k-means}

Algoritmus k-means využívá předchozí rovnosti a jediné, s~čím ``manipuluje'', jsou středy shluků. Nalezení globálního minima je NP-těžká úloha, proto se používá iterativní způsob, který hladově zmenšuje hodnotu účelové funkce.

\begin{algorithm}[H]
    \renewcommand{\thealgorithm}{}
    \caption{k-means}
    \begin{algorithmic}[1]
        \Require
        \Statex Množina dat $\D$ a požadovaný počet shluků $k$.
        \Statex Počáteční rozmístění $\bmu_1, \ldots, \bmu_k$.
        \Statex
        \While{hodnota účelové funkce ``hodně'' klesá}

        \State Roztřiď $\D$ do shluků
        \[
            Ci = \{ \x \in \D \mid i = \argmin_j \| \x - \bmu_j \| \}.
        \]

        \State Přepočítej středy nových clusterů
        \[
            \bmu_i \leftarrow \frac{1}{|C_i|} \sum_{\x \in C_i} \x.
        \]
        \EndWhile
    \end{algorithmic}
\end{algorithm}

Lze ukázat, že v~každém kroku algoritmu se hodnota účelové funkce zmenší, případně zůstane stejná (v~průběhu algoritmu je hodnota účelové funkce nerostoucí).

Běh algoritmu skončí, pokud se žádný ze středů nepřepočítá nebo pokud je rozdíl hodnot účelové funkce mezi iteracemi dostatečně malá.

Výsledek algoritmu hodně záleží na počátečním rozmístění středů, proto se typicky algoritmus spouští víckrát s~různými počátky, z~nichž si následně vybere shluk s~nejnižší účelovou funkcí. Existuje rozšíření algoritmu, které vybírá počátky tak, aby byly ``chytře'' rozmístěné (k-means++).

\paragraph{Volba počtu shluků}

k-means algoritmus dává různé výsledky pro různá $k$, proto je důležité jej stanovit dopředu. S~rostoucím $k$, zřejmě hodnota účelové funkce vždy klesá (případně neroste), proto je volba často subjektivní. Jednou z~metod (i když nespolehlivých) je metoda lokte (elbow method), která předpokládá to, že existuje nějaké ideální $k*$, takové, že hodnota účelové funkce s~rostoucím $k < k*$ rychle klesá, a pro $k > k*$ roste méně. Zlomový bod je předpokládaný loket, který ale není vždy úplně jednoznačný či optimální. Alternativní způsob ohodnocení clusteringu je například silhouette.

\subsection{Silhouette skóre}

Silhouette score je jednoduchou metodou pro evaluaci shlukování pro různá porovnání nebo i pro výběr optimálního počtu shluků (pro k-means).

Uvažujme nějaké shlukování $\D = C_1 \cup \ldots \cup C_k$ na metrickém prostoru $\calX$ s~metrikou $d(x, y)$, kde $\forall x \in \D: x \in C_{j(x)}$. Pro každý bod $x \in \D$ se spočítá:

\begin{itemize}

    \item průměrná vzdálenost bodu od ostatních bodů ve stejném shluku:
    \[
        a(x) = \frac{1}{|C_{j(x)}| - 1} \sum_{y \in C_{j(x)}, y \neq x} d(x, y)
    \]

    \item průměrná vzdálenost bodu od bodů v~jiném shluku:
    \[
        d(x, C_i) = \frac{1}{|C_{j(x)}|} \sum_{y \in C_i} d(x, y)
    \]

    \item průměrná vzdálenost bodu od nejbližšího jiného shluku:
    \[
        b(x) = \min_{i \neq j(x)} d(x, C_i)
    \]

\end{itemize}

\paragraph{Evaluace pomocí Silhouette skóre}

Finální skóre bodu $x \in \D$ se počítá následujícím vzorcem:
\[
    s(x) = \frac{b(x) - a(x)}{\max {a(x), b(x)}}
\]
V~případě jednoho shluku je $s(x) = 0$.

Pokud je $s(x)$ blízko 1, $b(x) \gg a(x)$, pak je bod dobře zařazen. Pokud je $s(x)$ blízko 0, $b(x) \approx a(x)$, je bod na kraji svého a sousedního shluku (také v~pořádku). Je-li však $s(x)$ blízko -1, $b(x) \ll a(x)$, je bod špatně zařazen. Poznamenejme, že v~takovém případě nestačí bod jen přemístit do druhého -- změnilo by celé ohodnocení (mohlo by být ve výsledku shlukování ještě zhoršit).

S~ohodnocením jednotlivých bodů lze nyní provést evaluaci jednotlivých clusterů i celého shlukování:
\[
    s(C_i) = \frac{1}{|C_i|} \sum_{x \in C_i} s(x)
    \qquad
    s = \frac{1}{|\D|} \sum_{x \in \D} s(x)
\]

Vyšší hodnota (blíže k~1) značí lepší shlukování (lepší celkové umístění jednotlivých bodů).
