\documentclass{article}
\title{AG1 - Cvičení V}
\author{Tommy Chu}
\date{}

\usepackage[czech]{babel}
\usepackage[utf8]{inputenc}
\usepackage[T1]{fontenc}
\usepackage{graphicx}
\usepackage{amsmath}
\usepackage{amsthm}
\usepackage{amssymb}
\usepackage{subfiles}
\usepackage{hyperref}
\usepackage{geometry}
\usepackage{mathtools}
\usepackage{algpseudocode}
\usepackage{algorithm}
\usepackage{physics}
\usepackage{dsfont}
\usepackage{bm}
\usepackage{bbm}
\usepackage{float}
\usepackage{enumitem}
\usepackage{multirow}
\usepackage{tikz}
\usepackage{tikz-cd}
\usepackage{pgfplots}
\usepackage{lmodern}
\usepackage{import}
\usepackage{microtype}
\usepackage{fancyhdr}
\usepackage{parskip}
\usepackage{mystyle}
\usepackage{soul}

\DeclareMathOperator{\E}{\mathbb{E}}

\begin{document}
\maketitle

\section*{Úloha 1}

\subsection*{Zadání}
Nechť ${(\Omega, P)}$ je diskrétní pravděpodobnostní prostor a nechť ${X \colon \Omega \rightarrow \mathbb{R}}$ je náhodná veličina. Ukažte, že existuje elementární jev ${\omega \in \Omega}$ takový, že ${X(\omega) \ge \E(X)}$.

\subsection*{Řešení}

Pro spor předpokládejme, že pro všechna ${\omega \in \Omega}$ platí ${X(\omega) < \E(X)}$.
\begin{align*}
    \E(X) & = \sum_{\omega \in \Omega} X(\omega) \cdot P(\omega)            \\
          & \stackrel{P.}{<} \sum_{\omega \in \Omega} \E(X) \cdot P(\omega) \\
          & = \E(X) \cdot \sum_{\omega \in \Omega} P(\omega)                \\
          & = \E(X) \cdot 1
\end{align*}

$\E(X) < \E(X)$ je zřejmě spor a původní předpoklad je tudíž nepravdivý.

\paragraph*{Závěr:} Pravdivost věty v zadání jsme potvrdili vyvrácením její negace.
\qed

\end{document}
