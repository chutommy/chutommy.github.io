\documentclass{article}
\title{BI-ZUM: Úlohy 2. cvičení}
\author{Tommy Chu}
\date{}

\usepackage[czech]{babel}
\usepackage[utf8]{inputenc}
\usepackage[T1]{fontenc}
\usepackage{amsmath}
\usepackage{amsthm}
\usepackage{amssymb}
\usepackage{parskip}
\usepackage{xcolor}
\usepackage{titling}
\usepackage{stmaryrd}
\usepackage{colortbl}

\usepackage[a4paper, total={150mm,257mm}]{geometry}
\pagenumbering{gobble}

\newcommand{\Nz}{\mathbb{N}_0}
\newcommand{\U}{\mathcal{U}}
\newcommand{\mS}{\text{S}}
\newcommand{\I}{\text{I}}
\newcommand{\G}{\text{G}}
\newcommand{\A}{\text{A}}
\newcommand{\myb}{{\color{gray} b}}
\newcommand{\mand}{\;{\color{gray} \land}\;}

\newcommand{\Pmin}{\text{P}_\text{min}}

\begin{document}
    \maketitle

    \section*{Úloha 1}

    \subsection*{Stavový prostor}

    Zavedeme množinu
    \(
    \U = \{ 1, 3, 5, 8, 12, b \},
    \)
    kde čísla představují osoby s~příslušnou dobou přesunu a~$b = 0$ baterku.
    Množinu stavů definujeme následovně:
    \[
        \mS = \big\{
        (D, t) \mid D \subseteq \U \mand t \in \Nz
        \big\},
    \]
    kde $D$ odpovídá osobám a baterce, které jsou v~čase $t$ u~domku, a~$\U \smallsetminus D$ jednoznačně určuje množinu osob na~kraji lesa.

    Přemístění osob z~kraje lesa k~domku zachycuje množina $\A_\text{k}$
    a v opačném směru množina $\A_\text{od}$:
    \[
        \begin{array}{llllll}
            \A_\text{k}
            & = \Big\{
            \big\{ (D, t), (D {\color{blue} \;\cup\; Q}, t + \max Q) \big\}
            & \mid
            & (D, t) \in \mS \mand \# Q \in \{2, 3\}
            \mand b \in Q \subseteq  {\color{blue} \U\smallsetminus D}
            & \Big\}

            \\[3mm]

            \A_\text{od}
            & = \Big\{
            \big\{ (D, t), (D {\color{blue} \;\smallsetminus\; Q}, t + \max Q) \big\}
            & \mid
            & (D, t) \in \mS \mand \# Q \in \{2, 3\}
            \mand b \in Q \subseteq {\color{blue} D}
            & \Big\}
        \end{array}
    \]
    Množina akcí je $\A = \A_\text{k} \cup \A_\text{od}$.

    Úlohu formulujeme jako prohledávání stavového prostoru $(S, A)$ s~počátečním stavem~$(\emptyset, 0)$ -- nikdo není u~domku, a množinou koncových stavů~$\{ (\U, t) \mid t \in \Nz \}$ -- všichni jsou u~domku.

    \subsection*{Řešení}

    Cesta splňující limit 30 minut je například:
    \begin{center}
    { \renewcommand{\arraystretch}{1.2}
        \begin{tabular}
        { r !{\color{lightgray}\vrule} c !{\color{lightgray}\vrule} r l !{\color{lightgray}\vrule} c }
            $t$ & $D$                        &               & $Q$               & $\U \smallsetminus D$      \\
            \hline
            0   & $\emptyset$                & $\leftarrow$  & $\{1,3, \myb\}$   & $\{1, 3, 5, 8, 12, \myb\}$ \\[2mm]
            3   & $\{1, 3, \myb\}$           & $\rightarrow$ & $\{1, \myb\}$     & $\{5, 8, 12\}$             \\[2mm]
            4   & $\{3\}$                    & $\leftarrow$  & $\{8, 12, \myb\}$ & $\{1, 5, 8, 12, \myb\}$    \\[2mm]
            16  & $\{3, 8, 12, \myb\}$       & $\rightarrow$ & $\{3, \myb\}$     & $\{1, 5\}$                 \\[2mm]
            19  & $\{8, 12\}$                & $\leftarrow$  & $\{1, 5, \myb\}$  & $\{1, 3, 5, \myb\}$        \\[2mm]
            24  & $\{1, 5, 8, 12, \myb\}$    & $\rightarrow$ & $\{1, \myb\}$     & $\{3\}$                    \\[2mm]
            25  & $\{5, 8, 12\}$             & $\leftarrow$  & $\{1,3, \myb\}$   & $\{1, 3, \myb\}$           \\[2mm]
            28  & $\{1, 3, 5, 8, 12, \myb\}$ &               &                   & $\emptyset$                \\[2mm]
        \end{tabular}}
    \end{center}

    \section*{Úloha 2}

    Existuje pouze jedna nejkratší cesta z~vrcholu 0 do vrcholu 3, kterou je
    \(
    \Pmin: (0 \rightarrow 4 \rightarrow 7 \rightarrow 3).
    \)

    DFS zvolí nejkratší cestu, právě když se v~prvním kroku rozhodne navštívit vrchol~4.
    V~opačném případě nalezne cestu přes vrcholy $(0 \rightarrow \cdots \rightarrow 6 \rightarrow 5 \rightarrow 7 \rightarrow 3)$,
    což je vždy cesta delší než $\Pmin$.

    Bez bližší specifikace implementace DFS, nelze určit, kterou cestu algoritmus zvolí. Může se například stát, že se zanoří $(0 \rightarrow 1 \rightarrow 6 \rightarrow 5 \rightarrow 7 \rightarrow 3)$. V~takový moment algoritmus cestu do vrcholu~3 nalezne a následně skončí. Nevrátí však nejkratší cestu ($\lightning$).

\end{document}
